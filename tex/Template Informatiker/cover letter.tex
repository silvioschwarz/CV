\documentclass[11pt,a4paper,roman]{moderncv}      
\usepackage[ngerman]{babel}



\moderncvstyle{classic}                            
\moderncvcolor{green}      
\definecolor{pblue}{HTML}{0395DE}
                      

% character encoding
\usepackage[utf8]{inputenc}                     

% adjust the page margins
\usepackage[scale=0.75]{geometry}

% personal data
\name{Silvio}{Schwarz}


\phone[mobile]{+49 174 6507598}               
\email{silvio{\_}schwarz@web.de}                             


\begin{document}
\recipient{~}{ }
\date{\today}
\opening{\textbf{Bewerbung Auszubildender zum Fachinformatiker}}
\closing{Ich hoffe, ich konnte Sie mit meiner Bewerbung überzeugen und freue mich auf ein persönliches Gespräch.\\
\vspace{1cm}
Mit freundlichen Grüßen,\\ \vspace{0.2cm}
\includegraphics[scale=0.4]{../../img/Unterschrift_Silvio} \vspace{-1cm}
}
\enclosure[Anhang]{Lebenslauf, Abiturzeugnis, Bachelorzeugnis, Leistungsnachweise Masterstudium, Zertifikate Onlinekurse}
\makelettertitle

% The first paragraph of your job application letter should include information on why you are writing. Mention the job you are applying for and where you found the position. If you have a contact at the company, mention the person's name and your connection here.
Sehr geehrte Dame oder Herr,\\
durch einen Jobnewsletter von baito.de bin ich auf Ihr Ausbildungsangebot zum Fachinformatiker aufmerksam geworden und freue mich, mich hiermit zu bewerben.

Ich sehe mich als einen geeigneten Kandidaten für diese Position, da ich durch mein Studium der Geowissenschaften verschieden Programmiersprachen wie Python und R bereits beherrsche. Darüber hinaus habe ich durch ein Praktikum im Herbst 2012 bei Wolfram|Alpha in den USA, sowie weiteren Anstellungen als studentischer Mitarbeiter eine umfangreiche Erfahrung als Programmierer und in der IT-Verwaltung.

Besonders möchte ich hierbei auf die Ergebnisse meiner Bachelorarbeit verweisen, die ich in einer \href{https://earthquake-distances.herokuapp.com/}{\color{pblue}\underline{interaktiven Webumgebung}} darstellen konnte und die Teilnahme an einem Hackathon des HPI (Hasso-Plattner-Institut) in 2019, in dessen Rahmen innerhalb von 22 Stunden eine \href{https://earthquake-distances.herokuapp.com}{\color{pblue}\underline{interaktive Karte}} bewaffneter Konflikte in Afrika entstanden ist.

Abschließend ist mir durch mein Studium, das Erstellen von hochwertigen wissenschaftlichen Arbeiten und Präsentationen vertraut und ich sehe mich durch  ehrenamtliche Tätigkeit in der Nachhilfe für Physik und Mathematik auch geeignet um Schulungen und Worksshops zu organisieren.\\

\vspace{1cm}


\makeletterclosing

\end{document}